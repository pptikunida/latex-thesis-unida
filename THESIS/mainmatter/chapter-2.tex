\chapter{LITERATURE REVIEW}
	\section{Theoretical Foundation}
		\subsection{teory 1}
		\setlength{\leftskip}{1.4cm}
		Modern software development is fundamentally guided by the principles of software engineering, which provide a systematic and disciplined approach to designing, developing, testing, and maintaining software systems. This structured methodology is essential for managing the complexity of large-scale projects, ensuring that the final product is reliable, efficient, and meets all specified requirements\footcite{sommerville2016}. Without adhering to these engineering principles, a project risks facing significant issues in scalability, maintainability, and overall quality, making it a critical theoretical cornerstone for any application development.

		\subsection{teory 2}
		Among the various architectural patterns available, the Model-View-Controller (MVC) stands out as a prevalent paradigm for organizing application logic, particularly in web development. The core strength of MVC lies in its separation of concerns, which divides the application into three interconnected components: the Model for data handling, the View for user interface presentation, and the Controller for managing user input and business logic\footcite{majeed2018}., This separation not only enhances code organization but also facilitates parallel development and simplifies future modifications, a concept effectively demonstrated in early implementations of content management systems\footcite{kristoko2009}.
		
		\subsection{teory 3}
		Beyond the internal architecture, the success of a software application is heavily dependent on its external qualities, primarily its functionality and usability. Functionality refers to the set of features and capabilities that the system provides to its users, while usability measures how easily and effectively users can interact with those features to achieve their goals\footcite{goodwin1987}. A system that is functionally powerful but difficult to use will likely face poor user adoption. Therefore, a user-centered design approach that prioritizes both of these aspects is crucial for creating applications that are not only powerful but also intuitive and satisfying to use.
		
		\subsection{teory 4}
		The discourse on the relationship between science and religion provides a vital theoretical framework for this research. The concept of integrating and interconnecting religious sciences with empirical sciences aims to bridge the perceived gap between revelation and reason, proposing a holistic approach to knowledge\footcite{muqoyyidin2014}. This perspective challenges the notion that culture and religion are merely peripheral influences on science, arguing instead that they can be integral components in the pursuit of knowledge, shaping both the questions asked and the interpretations drawn\footcite{muslih2010}.
		
		\subsection{teory 5}
		The contemporary digital era has ushered in a profound transformation in how knowledge is disseminated and consumed, a phenomenon described as the "onlinization" of scholarly discourse\footcite{maulana2021}. This theoretical context is particularly relevant to the study of religious texts, where the migration from printed books to digital platforms has influenced both the development and the public's engagement with contemporary interpretations\footcite{sihabussalam2024}. Understanding this digital shift is essential for analyzing the impact of new media on traditional fields of study and the new opportunities it creates for scholarly communication.
		
	\section{Previous Studies}
		\subsection{Studies 1}
		Previous research in the Indonesian context has extensively documented the practical application of the MVC architecture across various domains. Studies have shown its successful implementation in building adaptive online quiz systems for educational purposes, highlighting its flexibility\footcite{hidayat2012}. Furthermore, its use in developing institutional websites, such as for vocational high schools, demonstrates its reliability for managing academic information systems\footcite{wijaya2019}. The adoption of frameworks like CodeIgniter for e-commerce platforms further solidifies the pattern's standing as a go-to solution for structured web development in the region\footcite{suharsana2016}.
		
		\subsection{Studies 2}
		The development of e-commerce websites has been a popular subject of research, often utilizing modern PHP frameworks that inherently implement the MVC pattern. Multiple studies have showcased the use of Laravel for building online stores, from hijab shops to shoe stores, praising its robust features for creating complex and secure applications\footcite{aipina2022, nugroho2021}. These case studies often involve integrating third-party APIs, such as RajaOngkir for shipping cost calculation, to enhance the platform's functionality and provide a comprehensive user experience\footcite{prawito2020}.
			
		\subsection{Studies 3}
		The field of scientific Quranic exegesis in Indonesia has been explored from various epistemological standpoints. Research by Bakir (2016) analyzed the paradigm of contextualizing Quranic interpretation based on scientific findings, framing it within modern epistemology\footcite{bakir2016}. This approach aligns with broader scholarly efforts to interpret the Qur'an through the lens of modern science, a trend that has gained traction over the years\footcite{laila2014}. The integration of the Qur'an and science has become a distinct epistemological approach within Indonesian Islamic scholarship, aiming to affirm the scripture's relevance in the modern world\footcite{supriadi2018}.
				
		\subsection{Studies 4}
		The rise of digital platforms has created a new frontier for Quranic studies, a development captured by recent bibliometric research. These studies map the growth of Al-Qur'an interpretation research in the digital era, identifying key trends and contributors using computational methods\footcite{fadhilah2021}. The digital transformation is also evident in comparative studies of online tafsir platforms, which analyze how different websites present and structure Quranic knowledge for a digital audience\footcite{azizah2024}. This body of research underscores the irreversible shift towards digital mediums for religious scholarship.
					
		\subsection{Studies 5}
		On an international level, the scientific approach in Quranic exegesis has been a subject of continuous academic discussion, exploring its emergence and related issues\footcite{ismail2021}. Scholars have examined the contributions of various individuals, institutions, and even governments in promoting scientific readings of the Qur'an\footcite{asnawi2021}. This integration of scientific knowledge and religious sciences is not merely a theoretical exercise but also has practical implications for Islamic education, aiming to foster a holistic understanding among students\footcite{nasir2022}.