\chapter{LITERATURE REVIEW \& THEORETICAL FRAMEWORK}
	\section{Previous Studies}
	
	\begin{secenumerate}
		\item \lipsum[1]
		\item \lipsum[1]
		\item \lipsum[1]
	\end{secenumerate}

	\section{Theoretical Framework}
	\begin{secenumerate}
		\item concept al-qur'an
		\begin{secenumerate}
			\item \textbf{QS Fusshilat}
			\begin{otherlanguage*}{arabic}
				\noindent
				سَنُرِيهِمْ آيَاتِنَا فِي الْآفَاقِ وَفِي أَنْفُسِهِمْ حَتَّى يَتَبَيَّنَ لَهُمْ أَنَّهُ الْحَقُّ ۗ أَوَلَمْ يَكْفِ بِرَبِّكَ أَنَّهُ عَلَىٰ كُلِّ شَيْءٍ شَهِيدٌ
			\end{otherlanguage*}
			\vspace{0.5cm}
			
			\textit{\textbf{Meaning:}"We will show them Our signs in the horizons and within themselves until it becomes clear to them that it (the Qur'an) is the truth. Is it not enough (for you) that your Lord is a witness over all things?'" (Q.S. Fuṣṣilat 41:53)}\\
			
			\lipsum [3]
			
		\end{secenumerate}
		\item \textbf{qardh}
		\lipsum [3]
	\end{secenumerate}
	\section{Conceptual Framework}
	the first concept was like~\ref{fig:contoh-f}
	
	\begin{figure}[h!]
		\centering %position
		\includegraphics[width=0.8\textwidth]{assets/gambar-1.jpg}
		\caption{ini cuma contoh doang.}
		\label{fig:contoh-f} %UNIQUE COMMAND
	\end{figure}
